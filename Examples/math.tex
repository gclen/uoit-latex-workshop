\documentclass[a4paper]{article}
\usepackage{amsmath}

\begin{document}

\title{Using math mode}
\author{You \and Me}
\date{\today}
\maketitle

One of the primary reasons you will use \LaTeX \ is for writing equations.  

\begin{equation}
F_{net}=ma
\end{equation}

Equations don't have to be numbered. You can disable the numbering by putting an asterisk in the begin and end statements 
\begin{equation*}
E=mc^{2}
\end{equation*}
It is also very convenient to write multiple lines in an equation using the align environment:
\begin{align*}
2x - 5y &=  8 \\ 
3x + 9y &=  -12
\end{align*}
You can write all sorts of fancy symbols (which can be found on the cheatsheet!)
\begin{equation*}
i\hbar \frac{\partial | \Psi \rangle}{ \partial t}=\hat{H}|\Psi \rangle
\end{equation*}

Math mode can be used inline with text (e.g. $e^{-\lambda x}$) which is very convenient. All you need to do is wrap your equation (or whatever you are using) in dollar signs.  

\end{document}



